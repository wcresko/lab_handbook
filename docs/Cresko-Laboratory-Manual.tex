% Options for packages loaded elsewhere
\PassOptionsToPackage{unicode}{hyperref}
\PassOptionsToPackage{hyphens}{url}
%
\documentclass[
]{book}
\usepackage{amsmath,amssymb}
\usepackage{lmodern}
\usepackage{ifxetex,ifluatex}
\ifnum 0\ifxetex 1\fi\ifluatex 1\fi=0 % if pdftex
  \usepackage[T1]{fontenc}
  \usepackage[utf8]{inputenc}
  \usepackage{textcomp} % provide euro and other symbols
\else % if luatex or xetex
  \usepackage{unicode-math}
  \defaultfontfeatures{Scale=MatchLowercase}
  \defaultfontfeatures[\rmfamily]{Ligatures=TeX,Scale=1}
\fi
% Use upquote if available, for straight quotes in verbatim environments
\IfFileExists{upquote.sty}{\usepackage{upquote}}{}
\IfFileExists{microtype.sty}{% use microtype if available
  \usepackage[]{microtype}
  \UseMicrotypeSet[protrusion]{basicmath} % disable protrusion for tt fonts
}{}
\makeatletter
\@ifundefined{KOMAClassName}{% if non-KOMA class
  \IfFileExists{parskip.sty}{%
    \usepackage{parskip}
  }{% else
    \setlength{\parindent}{0pt}
    \setlength{\parskip}{6pt plus 2pt minus 1pt}}
}{% if KOMA class
  \KOMAoptions{parskip=half}}
\makeatother
\usepackage{xcolor}
\IfFileExists{xurl.sty}{\usepackage{xurl}}{} % add URL line breaks if available
\IfFileExists{bookmark.sty}{\usepackage{bookmark}}{\usepackage{hyperref}}
\hypersetup{
  pdftitle={Cresko Laboratory Manual},
  pdfauthor={Cresko Lab},
  hidelinks,
  pdfcreator={LaTeX via pandoc}}
\urlstyle{same} % disable monospaced font for URLs
\usepackage{longtable,booktabs,array}
\usepackage{calc} % for calculating minipage widths
% Correct order of tables after \paragraph or \subparagraph
\usepackage{etoolbox}
\makeatletter
\patchcmd\longtable{\par}{\if@noskipsec\mbox{}\fi\par}{}{}
\makeatother
% Allow footnotes in longtable head/foot
\IfFileExists{footnotehyper.sty}{\usepackage{footnotehyper}}{\usepackage{footnote}}
\makesavenoteenv{longtable}
\usepackage{graphicx}
\makeatletter
\def\maxwidth{\ifdim\Gin@nat@width>\linewidth\linewidth\else\Gin@nat@width\fi}
\def\maxheight{\ifdim\Gin@nat@height>\textheight\textheight\else\Gin@nat@height\fi}
\makeatother
% Scale images if necessary, so that they will not overflow the page
% margins by default, and it is still possible to overwrite the defaults
% using explicit options in \includegraphics[width, height, ...]{}
\setkeys{Gin}{width=\maxwidth,height=\maxheight,keepaspectratio}
% Set default figure placement to htbp
\makeatletter
\def\fps@figure{htbp}
\makeatother
\setlength{\emergencystretch}{3em} % prevent overfull lines
\providecommand{\tightlist}{%
  \setlength{\itemsep}{0pt}\setlength{\parskip}{0pt}}
\setcounter{secnumdepth}{5}
\usepackage{booktabs}
\usepackage{amsthm}
\makeatletter
\def\thm@space@setup{%
  \thm@preskip=8pt plus 2pt minus 4pt
  \thm@postskip=\thm@preskip
}
\makeatother
\ifluatex
  \usepackage{selnolig}  % disable illegal ligatures
\fi
\usepackage[]{natbib}
\bibliographystyle{apalike}

\title{Cresko Laboratory Manual}
\author{Cresko Lab}
\date{2021-04-23}

\begin{document}
\maketitle

{
\setcounter{tocdepth}{1}
\tableofcontents
}
Description of our laboratory

\hypertarget{introduction-to-the-lab}{%
\chapter{Introduction to the Lab}\label{introduction-to-the-lab}}

Description of our lab \ldots.

\hypertarget{mission-and-vision}{%
\chapter{Mission and Vision}\label{mission-and-vision}}

We describe our methods in this chapter.

\hypertarget{expectations-in-the-laboratory}{%
\chapter{Expectations in the Laboratory}\label{expectations-in-the-laboratory}}

We describe our methods in this chapter.

\hypertarget{iacuc}{%
\chapter{IACUC}\label{iacuc}}

descriptions of animal care IACUC protocols

\hypertarget{histological-protocols}{%
\chapter{Histological Protocols}\label{histological-protocols}}

\hypertarget{alizarin-staining}{%
\section{Alizarin Staining}\label{alizarin-staining}}

\hypertarget{purpose-alizarin-staining-of-fixed-adult-stickleback.}{%
\subsection{PURPOSE: Alizarin staining of fixed adult stickleback.}\label{purpose-alizarin-staining-of-fixed-adult-stickleback.}}

\hypertarget{materials}{%
\subsection{MATERIALS:}\label{materials}}

\begin{itemize}
\tightlist
\item
  0.5\% Alizarin red S Stock: To make 50 mls add 0.25g alizarin red S powder to 50 ml water.
\item
  0.025\% Alizarin Stain: To make 100 mls: Add 500µl 0.5\% alizarin red S (stock) to 99.5ml 1\% KOH
\item
  1 Liter: Add 5ml 0.5\% alizarin red S (stock) to 9950ml (1 liter) 1\%KOH
\item
  3\% H202/0.5\%KOH: Mix and keep at 4C; Before using, bring to room temperature to hold down
\item
  introducing bubbles under the skin: 0.5ml 6\%H202 \& 0.5ml 1\%KOH.
\item
  MESAB: Tricaine: 3-amino benzoic acid ethyl ester from Sigma (Cat \# A-5040). Mix in fish safe container with a stir bar:

  \begin{itemize}
  \tightlist
  \item
    400 mg tricaine powder
  \item
    800 mg Na2HPO4 (anhydrous)
  \item
    100 ml glass distilled water
  \end{itemize}
\end{itemize}

Adjust to \textasciitilde pH 7 with a drop at a time of 1N NaOH or 1N HCl if needed
but it's usually right if you weigh the sodium phosphate carefully and
measure the water with a graduated cylinder.

For storage: Aliquot into 6 x 25 ml fish safe plastic bottles and store
at 4C. Label with date made and use within a couple of weeks.

8\% PFA: \{\#pfa .subhead2\}

\begin{itemize}
\tightlist
\item
  8 g Pelleted PFA (Ted Pella, Inc.; cat\# 18501)
\item
  90 ml dH2O
\item
  25 drops 1N NaOH
\end{itemize}

\begin{enumerate}
\def\labelenumi{\arabic{enumi}.}
\tightlist
\item
  Heat at very low heat and stir until solution clears.
\item
  Add 25 drops 1N HCl. pH should be 7.0-7.2.
\item
  Filter and store at 4C not more than 1 week.
\item
  Use as 4\% PFA: dilute 1:1 with 2X PBS, do not store solution more
  than a few hours.
\end{enumerate}

2X PBS \{\#x-pbs .subhead2\}

\begin{itemize}
\tightlist
\item
  1.6\% NaCl
\item
  0.04\% KCl
\item
  0.04 M PO4 pH 7.0- 7.3
\end{itemize}

\hypertarget{procedure}{%
\subsection{Procedure:}\label{procedure}}

Day

Step

Time for Step

Date and Time

1

2h-8h at R/T depending on size on shaker.

1h or longer at R/T on shaker.

Without agitation and with lid open until eyes start to lighten and all
skin pigment is gone (usually about an hour or more)

2

2 h to O/N at R/T on shaker

2 h to overnight O/N at R/T on shaker.\\
Check for bone staining.

R/T on shaker until excess stain in tissue is gone.

Without agitation

Wild caught specimens are put in 100\% EtOH in the field and then
rehydrated and put into 4\% when back in the lab.

  \bibliography{book.bib,packages.bib}

\end{document}

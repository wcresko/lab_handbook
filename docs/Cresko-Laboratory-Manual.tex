% Options for packages loaded elsewhere
\PassOptionsToPackage{unicode}{hyperref}
\PassOptionsToPackage{hyphens}{url}
%
\documentclass[
]{book}
\usepackage{amsmath,amssymb}
\usepackage{lmodern}
\usepackage{ifxetex,ifluatex}
\ifnum 0\ifxetex 1\fi\ifluatex 1\fi=0 % if pdftex
  \usepackage[T1]{fontenc}
  \usepackage[utf8]{inputenc}
  \usepackage{textcomp} % provide euro and other symbols
\else % if luatex or xetex
  \usepackage{unicode-math}
  \defaultfontfeatures{Scale=MatchLowercase}
  \defaultfontfeatures[\rmfamily]{Ligatures=TeX,Scale=1}
\fi
% Use upquote if available, for straight quotes in verbatim environments
\IfFileExists{upquote.sty}{\usepackage{upquote}}{}
\IfFileExists{microtype.sty}{% use microtype if available
  \usepackage[]{microtype}
  \UseMicrotypeSet[protrusion]{basicmath} % disable protrusion for tt fonts
}{}
\makeatletter
\@ifundefined{KOMAClassName}{% if non-KOMA class
  \IfFileExists{parskip.sty}{%
    \usepackage{parskip}
  }{% else
    \setlength{\parindent}{0pt}
    \setlength{\parskip}{6pt plus 2pt minus 1pt}}
}{% if KOMA class
  \KOMAoptions{parskip=half}}
\makeatother
\usepackage{xcolor}
\IfFileExists{xurl.sty}{\usepackage{xurl}}{} % add URL line breaks if available
\IfFileExists{bookmark.sty}{\usepackage{bookmark}}{\usepackage{hyperref}}
\hypersetup{
  pdftitle={Cresko Laboratory Manual},
  pdfauthor={Cresko Lab},
  hidelinks,
  pdfcreator={LaTeX via pandoc}}
\urlstyle{same} % disable monospaced font for URLs
\usepackage{longtable,booktabs,array}
\usepackage{calc} % for calculating minipage widths
% Correct order of tables after \paragraph or \subparagraph
\usepackage{etoolbox}
\makeatletter
\patchcmd\longtable{\par}{\if@noskipsec\mbox{}\fi\par}{}{}
\makeatother
% Allow footnotes in longtable head/foot
\IfFileExists{footnotehyper.sty}{\usepackage{footnotehyper}}{\usepackage{footnote}}
\makesavenoteenv{longtable}
\usepackage{graphicx}
\makeatletter
\def\maxwidth{\ifdim\Gin@nat@width>\linewidth\linewidth\else\Gin@nat@width\fi}
\def\maxheight{\ifdim\Gin@nat@height>\textheight\textheight\else\Gin@nat@height\fi}
\makeatother
% Scale images if necessary, so that they will not overflow the page
% margins by default, and it is still possible to overwrite the defaults
% using explicit options in \includegraphics[width, height, ...]{}
\setkeys{Gin}{width=\maxwidth,height=\maxheight,keepaspectratio}
% Set default figure placement to htbp
\makeatletter
\def\fps@figure{htbp}
\makeatother
\setlength{\emergencystretch}{3em} % prevent overfull lines
\providecommand{\tightlist}{%
  \setlength{\itemsep}{0pt}\setlength{\parskip}{0pt}}
\setcounter{secnumdepth}{5}
\usepackage{booktabs}
\usepackage{amsthm}
\makeatletter
\def\thm@space@setup{%
  \thm@preskip=8pt plus 2pt minus 4pt
  \thm@postskip=\thm@preskip
}
\makeatother
\ifluatex
  \usepackage{selnolig}  % disable illegal ligatures
\fi
\usepackage[]{natbib}
\bibliographystyle{apalike}

\title{Cresko Laboratory Manual}
\author{Cresko Lab}
\date{2021-04-24}

\begin{document}
\maketitle

{
\setcounter{tocdepth}{1}
\tableofcontents
}
\hypertarget{the-cresko-lab}{%
\chapter{The Cresko Lab}\label{the-cresko-lab}}

Description of our laboratory

\hypertarget{introduction-to-the-lab}{%
\chapter{Introduction to the Lab}\label{introduction-to-the-lab}}

Description of our lab \ldots.

\hypertarget{mission-and-vision}{%
\chapter{Mission and Vision}\label{mission-and-vision}}

We describe our methods in this chapter.

\hypertarget{expectations-in-the-laboratory}{%
\chapter{Expectations in the Laboratory}\label{expectations-in-the-laboratory}}

\hypertarget{scientific-ethics-and-integrity}{%
\section{Scientific Ethics and Integrity}\label{scientific-ethics-and-integrity}}

\begin{itemize}
\tightlist
\item
  xx
\item
  xx
\item
  xx
\end{itemize}

\hypertarget{authorship-of-manuscripts}{%
\section{Authorship of Manuscripts}\label{authorship-of-manuscripts}}

\textbf{Recommended: At the start of each project, design your plan for authorship of the project so
everyone knows the expectations}

\emph{Authorship criteria}:

\begin{enumerate}
\def\labelenumi{\arabic{enumi})}
\tightlist
\item
  Makes a significant intellectual contribution to research ideas and experimental design
\end{enumerate}

OR

\begin{enumerate}
\def\labelenumi{\arabic{enumi})}
\setcounter{enumi}{1}
\tightlist
\item
  Makes a significant contribution to data acquisition, data generation, data analysis, data
  interpretation, research coordination, and/or financial support of research
\end{enumerate}

AND

\begin{enumerate}
\def\labelenumi{\arabic{enumi})}
\setcounter{enumi}{2}
\tightlist
\item
  Contributes to writing part of the manuscript, in addition to editing revisions before
  submission for publication
\end{enumerate}

AND

\begin{enumerate}
\def\labelenumi{\arabic{enumi})}
\setcounter{enumi}{3}
\tightlist
\item
  Remains involved throughout the submission and revision process until final publication
\end{enumerate}

*Research participants not meeting the criteria should be listed in the Acknowledgments
section of the final published manuscript

\emph{Authorship order}:

Generally, the person who had the most significant contribution to the project and who does
most of the writing will be the first author. In ecology, the last author is generally the PI of the
lab (although not always). The remaining authors are usually listed in their order of
contribution. However, if contributions were equivalent, then co-authors can be alphabetized
or ordered according to their time since involvement in the project.

\hypertarget{cresko-lab-safety-protocols}{%
\chapter{CRESKO LAB SAFETY PROTOCOLS}\label{cresko-lab-safety-protocols}}

\textbf{FOR YOUR OWN SAFETY AND THE SAFETY OF OTHERS, HEED THE FOLLOWING RULES!}

EMERGENCY CONTACT: dial 911 first, \emph{AND} 6-2919 (EHS) \textbar{} Mark Cell 541-505-0006

\textbf{•} **Safety Shower, Eyewash, Fire Extinguishers. Eyewashes must be flushed weekly. \emph{Undergraduate research assistants are responsible for flushing the safety showers each week.} \_Each lab member is responsible for knowing the locations of safety showers and fire extinguishers in the lab. Safety showers and fire extinguishers are tested annually by EHS.

\hypertarget{wear-a-lab-coat-and-closed-toed-shoes-when-working-with-the-following-chemicals}{%
\section{\texorpdfstring{Wear a lab coat** \textbf{and closed-toed shoes} \textbf{when working with the following chemicals:}}{Wear a lab coat** and closed-toed shoes when working with the following chemicals:}}\label{wear-a-lab-coat-and-closed-toed-shoes-when-working-with-the-following-chemicals}}

\begin{itemize}
\item
  organics (e.g.~phenol/chloroform, Trizol, DNAzol, formaldehyde, formamide, methanol)
\item
  strong acids and bases
\end{itemize}

\hypertarget{wear-eye-protection-when-working-with}{%
\section{\texorpdfstring{Wear** \textbf{eye protection} \textbf{when working with:}}{Wear** eye protection when working with:}}\label{wear-eye-protection-when-working-with}}

\begin{itemize}
\item
  UV light (UV opaque glasses/face shield)
\item
  phenol/chloroform, strong acids/bases, and any splash hazard with anything hazardous in it.
\end{itemize}

\hypertarget{wear-safety-gloves-when-working-with-any-of-the-reagents-above.}{%
\section{Wear safety gloves when working with ANY of the reagents above.}\label{wear-safety-gloves-when-working-with-any-of-the-reagents-above.}}

Heed the ``one glove rule'': remove one glove when moving between rooms to avoid touching doorknobs with a contaminated glove. Note that glove materials differ in their permeability to different reagents. Standard nitrile gloves are adequate for our lab's standard procedures. However, if you are planning experiments that involve more dangerous reagents, consult with Luke Sitts at EHS to select appropriate gloves.

\hypertarget{disposal-of-common-hazardous-reagents-ehs-disposal-6-3192}{%
\section{Disposal of common hazardous reagents (EHS DISPOSAL: 6-3192)**}\label{disposal-of-common-hazardous-reagents-ehs-disposal-6-3192}}

\begin{itemize}
\item
  E. coli plates and recombinant materials: autoclave buckets or EHS biohazard incineration boxes
\item
  E. coli flasks/liquids: bleach, rinse, drain
\item
  Used alcohols, formaldehyde, and kit waste: waste containers under the thermocyclers.
\item
  organic solvents: waste bottles in hood.
\end{itemize}

\hypertarget{storage-of-hazardous-liquids}{%
\section{\texorpdfstring{Storage \textbf{of Hazardous Liquids}}{Storage of Hazardous Liquids}}\label{storage-of-hazardous-liquids}}

\begin{itemize}
\tightlist
\item
  Store flammables and strong acids in a latched METAL SAFETY CABINET UNDER THE HOOD.
\end{itemize}

\hypertarget{heating-liquids-in-the-microwave-oven}{%
\section{Heating Liquids in the Microwave Oven**}\label{heating-liquids-in-the-microwave-oven}}

Triple check that the cap is \emph{very} loose or (better) remove it entirely. Remelting of gels with DNA binding dyes is forbidden.

\hypertarget{bunsen-burners}{%
\section{Bunsen Burners**}\label{bunsen-burners}}

\begin{itemize}
\item
  Triple check that the gas is shut completely off before you leave the bench/ hood.
\item
  keep burners far away from any flammable liquids.
\end{itemize}

\hypertarget{liquid-nitrogen-and-dry-ice}{%
\section{Liquid Nitrogen and Dry Ice**}\label{liquid-nitrogen-and-dry-ice}}

\begin{itemize}
\item
  Use only in well ventilated spaces to avoid asphyxiation.
\item
  Never store in sealed containers to avoid explosions
\item
  Wear lab coat, gloves, goggles. In case of frostbite or burn, soak affected part in tepid water, seek medical attention
\end{itemize}

\hypertarget{iacuc}{%
\chapter{IACUC}\label{iacuc}}

descriptions of animal care IACUC protocols

\hypertarget{pipefish-husbandry-protocols}{%
\chapter{Pipefish Husbandry Protocols}\label{pipefish-husbandry-protocols}}

\hypertarget{pipefish-feeding}{%
\section{Pipefish Feeding}\label{pipefish-feeding}}

(created by M Currey 7/23/09)

\textbf{Materials Needed:}

\begin{itemize}
\item
  Decapsualted Brine Shrimp (see artemia decapsulations SOP)
\item
  Adult Brine Shrimp
\item
  Live Moina
\item
  Frozen myisid Shrimp
\item
  Live mysid shrimp
\item
  Shrimp collector
\item
  Squirt Bottle
\item ~
  \hypertarget{fish-foods-for-fry-juvenile-and-adults}{%
  \subsection{Fish foods for fry, juvenile and adults:}\label{fish-foods-for-fry-juvenile-and-adults}}
\item
  Fry - newly hatched baby brine shrimp (see hatching brine shrimp SOP), salt water copepods. Fry are fed once per day
\item
  Adult -- newly hatched brine shrimp, Adult brine shrimp, Moina. Adults are fed once per day. Feed adult brine shrimp when we have them. Use moina when we are out of adult brine shrimp. Adult brine shrimp are from a local fish store and are only available every tow weeks. They last \textasciitilde{} one week and therefore adult pipefish are fed adult brine shrimp for one week and moina the next.
\end{itemize}

\textbf{Fry:}

\begin{verbatim}
     Fry tanks are designated with an orange dot. 
\end{verbatim}

\begin{enumerate}
\def\labelenumi{\arabic{enumi}.}
\tightlist
\item
  Newly hatched brine: Collect newly hatched brine and place into a squirt bottle (see brine shrimp SOP). Feed all tanks with an orange dot.
\end{enumerate}

\textbf{Adults:}

\begin{verbatim}
    Adult tanks are designated with a yellow dot. 
\end{verbatim}

\begin{enumerate}
\def\labelenumi{\arabic{enumi}.}
\tightlist
\item
  Newly hatched brine: Collect newly hatched brine and place into a squirt bottle (see brine shrimp SOP). Feed all tanks with an orange dot.
\item
  Frozen Mysis: Obtain a quarter-sized piece of frozen mysis from the freezer. Place into squirt bottle and add water. Wait until mysis thaws and feed to all adult tanks.
\item
  Adult Brine shrimp: Scoop out adult brine shrimp with net. Wash into a ball and place over the top of squirt bottle. Wash ball of brine into squirt bottle and feed all adult pipefish.
\item
  Moina: Scoop out with net and wash into a ball. Invert ball over collection beaker and wash moina into beaker. Pour moina into squirt bottle and feed.
\item
  Live Mysid: See live foods SOP
\end{enumerate}

\hypertarget{live-food-culture-monia-and-mysid-shrimp}{%
\section{\texorpdfstring{\textbf{Live Food Culture, Monia and Mysid Shrimp:}}{Live Food Culture, Monia and Mysid Shrimp:}}\label{live-food-culture-monia-and-mysid-shrimp}}

\textbf{\emph{Moina}}

\textbf{Materials:}

\begin{itemize}
\tightlist
\item
  10 gallon glass tanks
\item
  corner sponge filter
\item
  Air supply
\item
  Rotifer diet
\item
  Powdered nannochloropsis
\end{itemize}

\textbf{Procedure:}

\begin{itemize}
\tightlist
\item
  Fill 10 gallon tank 3/4 full of stickleback system water
\item
  Add corner filter and activate with air.
\item
  Add Moina
\item
  Change water once every 2-3 weeks by removing half of the water and replacing with stickleback water.
\item
  DO NOT break tank down and clean as moina do not respond well to this.
\end{itemize}

\textbf{Feeding:}

\begin{itemize}
\tightlist
\item
  Add 15 drops of rotifer diet and 1/8 scoop of powdered nannochloropsis each day.
\end{itemize}

\hypertarget{collection-and-feeding-to-fish}{%
\subparagraph{Collection and feeding to fish}\label{collection-and-feeding-to-fish}}

\begin{itemize}
\tightlist
\item
  See pipefish feeding SOP
\end{itemize}

\hypertarget{mysid-shrimp}{%
\subparagraph{\texorpdfstring{\emph{Mysid Shrimp}}{Mysid Shrimp}}\label{mysid-shrimp}}

For a description of the mysid generator please visit:

\url{http://www.mblaquaculture.com/assets/docs/MBL_AQ_Mysid_Generator.pdf}

\textbf{Materials:}

\begin{itemize}
\tightlist
\item
  10 gallon tank generator system
\item
  Salt water
\end{itemize}

\textbf{Feeding:}

\begin{itemize}
\tightlist
\item
  Feed newly hatched brine shrimp daily to both adults and juveniles.
\end{itemize}

\textbf{Water Change:}

\begin{itemize}
\tightlist
\item
  2-3 times per week empty 5 gallons of water from the system and replace with new make up water.
\item
  Make new water in 5 gallon bucket by adding DI water and 2 scoops of salt.
\end{itemize}

\textbf{Juvenile Collection (Daily):}

\begin{itemize}
\tightlist
\item
  Turn off water to tanks.
\item
  Remove collection cup, using mysid system water, rinse juveniles into plastic container.
\item
  Pour juveniles into grow out tank.
\item
  Replace collection cup.
\item
  Turn water on and start siphon.
\end{itemize}

\textbf{Adults collection and feeding to pipefish:}

Juvenile will reach adult size in three weeks. At three weeks these new adults will replace old breeding adults. The old breeding adults that are being replaced are feed to the pipefish.

\begin{itemize}
\tightlist
\item
  Let juveniles grow to three weeks at which point they reach adult stage
\item
  Siphon adults through a net and collect in a container.
\item
  Siphon old adults out of one of the 10 gallon tanks and feed to pipefish
\item
  Clean tank, fill with water and add new adult.
\end{itemize}

\hypertarget{live-food-culture-monia-and-mysid-shrimp-1}{%
\section{Live Food Culture, Monia and Mysid Shrimp:}\label{live-food-culture-monia-and-mysid-shrimp-1}}

\textbf{\emph{Moina}}

\textbf{Materials:}

\begin{itemize}
\tightlist
\item
  10 gallon glass tanks
\item
  corner sponge filter
\item
  Air supply
\item
  Rotifer diet
\item
  Powdered nannochloropsis
\end{itemize}

\textbf{Procedure:}

\begin{itemize}
\tightlist
\item
  Fill 10 gallon tank 3/4 full of stickleback system water
\item
  Add corner filter and activate with air.
\item
  Add Moina
\item
  Change water once every 2-3 weeks by removing half of the water and replacing with stickleback water.
\item
  DO NOT break tank down and clean as moina do not respond well to this.
\end{itemize}

\textbf{Feeding:}

\begin{itemize}
\tightlist
\item
  Add 15 drops of rotifer diet and 1/8 scoop of powdered nannochloropsis each day.
\end{itemize}

\hypertarget{collection-and-feeding-to-fish-1}{%
\subparagraph{Collection and feeding to fish}\label{collection-and-feeding-to-fish-1}}

\begin{itemize}
\tightlist
\item
  See pipefish feeding SOP
\end{itemize}

\hypertarget{mysid-shrimp-1}{%
\subparagraph{\texorpdfstring{\emph{Mysid Shrimp}}{Mysid Shrimp}}\label{mysid-shrimp-1}}

For a description of the mysid generator please visit:

\url{http://www.mblaquaculture.com/assets/docs/MBL_AQ_Mysid_Generator.pdf}

\textbf{Materials:}

\begin{itemize}
\tightlist
\item
  10 gallon tank generator system
\item
  Salt water
\end{itemize}

\textbf{Feeding:}

\begin{itemize}
\tightlist
\item
  Feed newly hatched brine shrimp daily to both adults and juveniles.
\end{itemize}

\textbf{Water Change:}

\begin{itemize}
\tightlist
\item
  2-3 times per week empty 5 gallons of water from the system and replace with new make up water.
\item
  Make new water in 5 gallon bucket by adding DI water and 2 scoops of salt.
\end{itemize}

\textbf{Juvenile Collection (Daily):}

\begin{itemize}
\tightlist
\item
  Turn off water to tanks.
\item
  Remove collection cup, using mysid system water, rinse juveniles into plastic container.
\item
  Pour juveniles into grow out tank.
\item
  Replace collection cup.
\item
  Turn water on and start siphon.
\end{itemize}

\textbf{Adults collection and feeding to pipefish:}

Juvenile will reach adult size in three weeks. At three weeks these new adults will replace old breeding adults. The old breeding adults that are being replaced are feed to the pipefish.

\begin{itemize}
\tightlist
\item
  Let juveniles grow to three weeks at which point they reach adult stage
\item
  Siphon adults through a net and collect in a container.
\item
  Siphon old adults out of one of the 10 gallon tanks and feed to pipefish
\item
  Clean tank, fill with water and add new adult.
\end{itemize}

\hypertarget{histological-protocols}{%
\chapter{Histological Protocols}\label{histological-protocols}}

\hypertarget{alizarin-staining}{%
\section{Alizarin Staining}\label{alizarin-staining}}

\hypertarget{purpose-alizarin-staining-of-fixed-adult-stickleback.}{%
\subsection{PURPOSE: Alizarin staining of fixed adult stickleback.}\label{purpose-alizarin-staining-of-fixed-adult-stickleback.}}

\hypertarget{materials}{%
\subsection{MATERIALS:}\label{materials}}

\begin{itemize}
\tightlist
\item
  0.5\% Alizarin red S Stock: To make 50 mls add 0.25g alizarin red S powder to 50 ml water.
\item
  0.025\% Alizarin Stain: To make 100 mls: Add 500µl 0.5\% alizarin red S (stock) to 99.5ml 1\% KOH
\item
  1 Liter: Add 5ml 0.5\% alizarin red S (stock) to 9950ml (1 liter) 1\%KOH
\item
  3\% H202/0.5\%KOH: Mix and keep at 4C; Before using, bring to room temperature to hold down
\item
  introducing bubbles under the skin: 0.5ml 6\%H202 \& 0.5ml 1\%KOH.
\item
  MESAB: Tricaine: 3-amino benzoic acid ethyl ester from Sigma (Cat \# A-5040). Mix in fish safe container with a stir bar:

  \begin{itemize}
  \tightlist
  \item
    400 mg tricaine powder
  \item
    800 mg Na2HPO4 (anhydrous)
  \item
    100 ml glass distilled water
  \end{itemize}
\end{itemize}

Adjust to \textasciitilde pH 7 with a drop at a time of 1N NaOH or 1N HCl if needed
but it's usually right if you weigh the sodium phosphate carefully and
measure the water with a graduated cylinder.

For storage: Aliquot into 6 x 25 ml fish safe plastic bottles and store
at 4C. Label with date made and use within a couple of weeks.

8\% PFA: \{\#pfa .subhead2\}

\begin{itemize}
\tightlist
\item
  8 g Pelleted PFA (Ted Pella, Inc.; cat\# 18501)
\item
  90 ml dH2O
\item
  25 drops 1N NaOH
\end{itemize}

\begin{enumerate}
\def\labelenumi{\arabic{enumi}.}
\tightlist
\item
  Heat at very low heat and stir until solution clears.
\item
  Add 25 drops 1N HCl. pH should be 7.0-7.2.
\item
  Filter and store at 4C not more than 1 week.
\item
  Use as 4\% PFA: dilute 1:1 with 2X PBS, do not store solution more
  than a few hours.
\end{enumerate}

2X PBS \{\#x-pbs .subhead2\}

\begin{itemize}
\tightlist
\item
  1.6\% NaCl
\item
  0.04\% KCl
\item
  0.04 M PO4 pH 7.0- 7.3
\end{itemize}

\hypertarget{procedure}{%
\subsection{Procedure:}\label{procedure}}

Day

Step

Time for Step

Date and Time

1

2h-8h at R/T depending on size on shaker.

1h or longer at R/T on shaker.

Without agitation and with lid open until eyes start to lighten and all
skin pigment is gone (usually about an hour or more)

2

2 h to O/N at R/T on shaker

2 h to overnight O/N at R/T on shaker.\\
Check for bone staining.

R/T on shaker until excess stain in tissue is gone.

Without agitation

Wild caught specimens are put in 100\% EtOH in the field and then
rehydrated and put into 4\% when back in the lab.

  \bibliography{book.bib,packages.bib}

\end{document}
